\section{Solution}

\subsection{Overall architecture}

\begin{itemize}
    \item \textbf{EmbeddedSystemX} Is the overall class which implementes all functionality of the application
    any user interaction is done through a simple text based user interface written in main.
\end{itemize}

To avoid unncessary C++ semantics, the entire composition of EmbeddedSystemX and all its subcomponents
is written in one large \textit{.cpp} file, this is has mainly been done to reduce compiler complexity and
and practical implementation difficulties, which is bound to arise, due to things such as \textit{circular-referencing}. 
\\
\subsubsection{State Diagram} 
The overall structure follows the provided model of the system

\begin{figure}[!htb]
    \centering
    \includegraphics[width=0.4\textwidth]{pictures/state_machine.png}
    \caption{Overall statemachine of EmbeddedSystemX}
    \label{fig:overall_statemachine}
\end{figure}

There is a focus on the inner statemachine inside of the \textit{Operational} class, which contains a nested statemachien
In this case referred to as \textit{opFsm - operational state machine}.
\\
\subsubsection{Class Diagram}
To comform with the requirement of the GOF state machine, all of the states are implemented as classes, which are 
inheriting from the \textit{State}-class

\clearpage
\begin{figure}[!htb]
    \centering
    \includegraphics[width=0.5\textwidth]{pictures/class_diagram.png}
    \caption{The State class, is a pure virtual class, which will be explained in a later section}
    \label{fig:class_diagram}
\end{figure}

The states are implemented this way, so that the statemachine trusts the state class with all
the behavior of the implemented classes, this means that they all have the some core
functions: \textit{eventHandler(), onEntry(), onExit()} and so on. It only fits together with the 
the singleton pattern. So matter what iteration of the state class, there only exists
one instance of that class, that has the same public interface as the previous instance.

\subsubsection{Sequence Diagram}
the sequence of execution in the overall state machine is as follows
\begin{figure}[!htb]
    \centering
    \includegraphics[width=0.5\textwidth]{pictures/sequence_embeddedSysX.png}
    \caption{The initial phase of the EmbeddedSystemX's statemachine, here it automatically
    executes through the initial states. A small chance has been added so that the SelfTest
    might fail. Else it justs continues to the ready state. It also shows the lifetime of the
    states as objects.}
    \label{fig:sequence_embedX}
\end{figure}

The operational class is where the internal statemachine gets activated by the \textit{onEntry()} 
function inside the operational class. That signifies the start of the operational sequence diagram

\begin{figure}
    \centering
    \includegraphics[width=0.5\textwidth]{pictures/sequence_operational.png}
    \caption{The internal statemachine \textit{opFsm} inside operational, this is a concurrent
    statemachine to the on outside, however the two don't have any interplay.
    As with the outside statemachine, this also have classes as states, implemented by the 
    state/eventHandler mechanism, all are singletons aswell. All data passed through opFsm, are 
    referenced to the outside machine.}
    \label{fig:sequence_operational}
\end{figure}

The important thing in this sequence diagram here is, that the \textit{RealTimeLoop} is an 
active object, meaning that its method invocation has been decoupled from its exection, this
is also shown in figure: \ref{fig:overall_statemachine}. Where to overall mechanism follow
the following class diagram.

\begin{figure}
    \centering
    \includegraphics[width=0.5\textwidth]{pictures/state_active_object.png}
    \caption{The internal class diagram of the active object, which resides inside the 
    RealTimeLoop class}
    \label{fig:class_active_object}
\end{figure}